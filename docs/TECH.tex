\documentclass[11pt,a4paper]{article}

% 中文支持
\usepackage[UTF8]{ctex}

% 字体设置
% Overleaf 使用 XeLaTeX 编译,ctex 会自动配置中文字体
% 如需自定义等宽字体,取消下面注释:
% \usepackage{fontspec}
% \setmonofont{Fira Mono}[Scale=0.85]

% 页面设置
\usepackage[margin=2.5cm]{geometry}

% 图形和颜色
\usepackage{tikz}
\usepackage{xcolor}
\usetikzlibrary{shapes.geometric, arrows.meta, positioning, calc}

% 表格
\usepackage{booktabs}
\usepackage{array}
\usepackage{tabularx}
\usepackage{multirow}

% 代码高亮
\usepackage{listings}

% 超链接
\usepackage{hyperref}
\hypersetup{
    colorlinks=true,
    linkcolor=blue!60!black,
    urlcolor=blue!60!black
}

% 其他
\usepackage{enumitem}
\usepackage{fancyhdr}
\usepackage{titlesec}

% 颜色定义
\definecolor{codegreen}{rgb}{0,0.6,0}
\definecolor{codegray}{rgb}{0.5,0.5,0.5}
\definecolor{codepurple}{rgb}{0.58,0,0.82}
\definecolor{backcolour}{rgb}{0.97,0.97,0.97}
\definecolor{diverblue}{RGB}{41,98,255}
\definecolor{diverorange}{RGB}{255,152,0}

% 代码样式
\lstdefinestyle{mystyle}{
    backgroundcolor=\color{backcolour},
    commentstyle=\color{codegreen},
    keywordstyle=\color{blue},
    numberstyle=\tiny\color{codegray},
    stringstyle=\color{codepurple},
    basicstyle=\ttfamily\normalsize,
    breakatwhitespace=false,
    breaklines=true,
    captionpos=b,
    keepspaces=true,
    numbers=left,
    numbersep=8pt,
    showspaces=false,
    showstringspaces=false,
    showtabs=false,
    tabsize=4,
    frame=single,
    framerule=0.5pt,
    rulecolor=\color{gray!30},
    columns=fullflexible,        % 防止字符挤压
    basewidth=0.5em,             % 字符基准宽度
    lineskip=2pt,                % 行间距
    aboveskip=1em,               % 代码块上方间距
    belowskip=1em,               % 代码块下方间距
    xleftmargin=2em,             % 左边距
    framexleftmargin=1.5em       % 边框左边距
}
\lstset{style=mystyle}

% 标题样式
\titleformat{\section}{\Large\bfseries\color{diverblue}}{\thesection}{1em}{}
\titleformat{\subsection}{\large\bfseries}{\thesubsection}{1em}{}

% 页眉页脚
\pagestyle{fancy}
\fancyhf{}
\rhead{\small DIVER 技术白皮书}
\lhead{\small \leftmark}
\cfoot{\thepage}

\begin{document}

% 标题页
\begin{titlepage}
    \centering
    \vspace*{3cm}
    
    {\Huge\bfseries\color{diverblue} DIVER}\\[0.5cm]
    {\Large\color{gray} Dotnet Integrated Vehicle Embedded Runtime}\\[2cm]
    
    {\LARGE\bfseries 技术白皮书}\\[0.5cm]
    {\large 用 C\# 重新定义嵌入式开发}\\[3cm]
    
    \begin{tikzpicture}
        \node[draw=diverblue, line width=2pt, rounded corners=10pt, 
              minimum width=10cm, minimum height=3cm, 
              fill=diverblue!5] (box) {};
        \node at (box.center) {\large\textbf{让嵌入式开发像写应用程序一样简单}};
    \end{tikzpicture}
    
    \vfill
    {\large \today}
\end{titlepage}

\tableofcontents
\newpage

%========================================
\section{概述}
%========================================

\textbf{DIVER (Dotnet Integrated Vehicle Embedded Runtime)} 是一款革命性的嵌入式开发框架,首次实现了用 C\# 高级语言直接编写 MCU(微控制器)程序。通过独创的"末梢神经"架构,DIVER 在保持工业级实时性能的同时,大幅提升开发效率。

\vspace{1em}
\noindent\textbf{核心特性:}
\begin{itemize}[noitemsep]
    \item C\# 到 MCU 字节码编译 —— 用熟悉的语法编写嵌入式逻辑
    \item 类 RTOS 运行时 —— 可配置扫描间隔的确定性执行
    \item 双向数据交换 —— UpperIO(主机→MCU)和 LowerIO(MCU→主机)
    \item 硬件 IO 抽象 —— 支持 CAN、串口、GPIO、Modbus
    \item PC 端模拟调试 —— 无需硬件即可开发测试
    \item 源码级错误定位 —— MCU 崩溃时精确定位到 C\# 源代码行号
\end{itemize}

%========================================
\section{行业痛点}
%========================================

\subsection{嵌入式开发现状}

当前嵌入式开发面临诸多挑战,严重制约了开发效率和产品迭代速度:

\begin{table}[htbp]
\centering
\renewcommand{\arraystretch}{1.3}
\begin{tabularx}{\textwidth}{>{\bfseries}l X}
\toprule
痛点 & 影响 \\
\midrule
开发语言门槛高 & C/C++ 语法复杂,指针操作易出错,内存管理困难 \\
调试效率低下 & MCU 调试需要专用硬件(JTAG/SWD),问题定位困难 \\
开发周期漫长 & 从原型到稳定版本迭代周期长 \\
代码复用率低 & 不同 MCU 平台需要重写,知识资产难以沉淀 \\
人才门槛高 & 需要同时掌握硬件和底层软件知识 \\
\bottomrule
\end{tabularx}
\caption{嵌入式开发痛点分析}
\end{table}

\subsection{典型场景困境}

\begin{itemize}
    \item \textbf{机器人开发}:多电机协调控制逻辑复杂,调试一个 CAN 通信问题可能耗费数天
    \item \textbf{智能设备}:快速迭代需求与嵌入式开发周期矛盾
    \item \textbf{教育培训}:学生难以快速上手实时控制系统开发
\end{itemize}

%========================================
\section{DIVER 解决方案}
%========================================

\subsection{核心创新:C\# 编译到 MCU}

DIVER 实现了从传统嵌入式开发到现代化开发的跨越:

\begin{table}[htbp]
\centering
\renewcommand{\arraystretch}{1.3}
\begin{tabularx}{\textwidth}{>{\bfseries}l c X}
\toprule
对比维度 & 传统方式 & DIVER 方式 \\
\midrule
编程语言 & C/C++ & C\# \\
调试方式 & JTAG 硬件调试 & PC 端模拟调试 \\
内存管理 & 手动管理 & 自动管理 \\
代码移植 & 平台相关代码 & 跨平台通用代码 \\
开发工具 & 分散的工具链 & 统一的 Web IDE \\
\bottomrule
\end{tabularx}
\caption{传统开发与 DIVER 对比}
\end{table}

\subsection{"末梢神经"分层架构}

DIVER 采用创新的分层架构,将系统分为主机层和 MCU 层:

\begin{figure}[htbp]
\centering
\begin{tikzpicture}[
    box/.style={draw, rounded corners, minimum width=12cm, minimum height=2.5cm, align=center},
    arrow/.style={-{Stealth[length=3mm]}, line width=1.5pt}
]
    % 主机层
    \node[box, fill=diverblue!15, draw=diverblue, line width=1.5pt] (host) at (0,3) {
        \textbf{\large 主机层(大脑)}\\[0.3em]
        \begin{tabular}{cccc}
        复杂业务逻辑 & AI/ML 算法 & 数据分析 & 用户界面 \\
        .NET 生态系统 & 云端连接 & 日志记录 & 可视化
        \end{tabular}
    };
    
    % MCU层
    \node[box, fill=diverorange!15, draw=diverorange, line width=1.5pt] (mcu) at (0,-0.5) {
        \textbf{\large MCU 层(末梢神经)}\\[0.3em]
        \begin{tabular}{cccc}
        实时 IO 控制 & 确定性时序 & CAN/Modbus & GPIO/ADC \\
        可配置扫描周期 & 硬实时响应 & 安全保护 & 电机控制
        \end{tabular}
    };
    
    % 双向箭头
    \draw[arrow, diverblue] ($(host.south)-(1.5,0)$) -- node[left, font=\small] {UpperIO} ($(mcu.north)-(1.5,0)$);
    \draw[arrow, diverorange] ($(mcu.north)+(1.5,0)$) -- node[right, font=\small] {LowerIO} ($(host.south)+(1.5,0)$);
    
    % 通信标签
    \node[fill=white, font=\small\bfseries] at (0,1.25) {串口 / 以太网};
    
\end{tikzpicture}
\caption{DIVER "末梢神经"分层架构}
\end{figure}

\noindent\textbf{设计原则:}
\begin{itemize}[noitemsep]
    \item \textbf{MCU 负责}:实时 IO 处理、基本控制循环、确定性时序
    \item \textbf{主机负责}:复杂逻辑、数据记录、用户界面、机器学习
    \item \textbf{清晰边界}:防止时序问题影响实时操作
\end{itemize}

\subsection{双向数据流}

DIVER 通过属性标注实现清晰的数据流向定义:

\begin{table}[htbp]
\centering
\renewcommand{\arraystretch}{1.3}
\begin{tabularx}{\textwidth}{l l X l}
\toprule
\textbf{方向} & \textbf{属性标注} & \textbf{用途} & \textbf{示例} \\
\midrule
主机 → MCU & \texttt{[AsUpperIO]} & 控制命令、设定值 & \texttt{motorSpeed} \\
MCU → 主机 & \texttt{[AsLowerIO]} & 传感器数据、状态 & \texttt{actualSpeed} \\
\bottomrule
\end{tabularx}
\caption{双向数据流定义}
\end{table}

%========================================
\section{核心技术特性}
%========================================

\subsection{DiverCompiler —— C\# 到字节码编译器}

基于 Fody IL Weaver 技术,DiverCompiler 将 C\# 代码编译为轻量级字节码:

\begin{lstlisting}[language={[Sharp]C}, caption={MCU 逻辑代码示例}]
// 开发者编写的 C# 代码
[LogicRunOnMCU(scanInterval = 50)]  // 50ms 扫描周期
public class MotorControl : LadderLogic<MyVehicle>
{
    public override void Operation(int iteration)
    {
        if (cart.sensorValue > 100)
            cart.motorSpeed = 0;
        else
            cart.motorSpeed = cart.targetSpeed;
    }
}
\end{lstlisting}

\noindent 编译器自动完成:
\begin{itemize}[noitemsep]
    \item 提取 MCU 逻辑代码
    \item 转换为紧凑字节码(\texttt{.bin})
    \item 生成源码映射(\texttt{.diver.map.json})
    \item 输出变量描述符(\texttt{.bin.json})
\end{itemize}

\subsection{MCU Runtime —— 轻量级虚拟机}

纯 C 实现的高效运行时,具备以下特点:

\begin{table}[htbp]
\centering
\renewcommand{\arraystretch}{1.2}
\begin{tabular}{ll}
\toprule
\textbf{特性} & \textbf{规格} \\
\midrule
Flash 占用 & < 32KB \\
RAM 占用 & < 8KB \\
执行模式 & 确定性扫描周期,无 GC 停顿 \\
IO 接口 & 统一的 Event/Stream/Snapshot 抽象 \\
\bottomrule
\end{tabular}
\caption{MCU Runtime 技术规格}
\end{table}

\begin{lstlisting}[language=C, caption={MCU 端执行循环}]
// MCU 端执行循环
while (1) {
    vm_put_snapshot_buffer(gpio_states);    // 输入 GPIO
    vm_put_event_buffer(can_messages);      // 输入 CAN
    vm_put_upper_memory(host_commands);     // 输入主机命令
    
    vm_run(iteration++);                    // 执行字节码
    
    send_to_host(vm_get_lower_memory());    // 输出状态
}
\end{lstlisting}

\subsection{硬件 IO 抽象层}

DIVER 提供三种类型的 IO 抽象,覆盖常见硬件接口:

\begin{table}[htbp]
\centering
\renewcommand{\arraystretch}{1.3}
\begin{tabularx}{\textwidth}{l l X}
\toprule
\textbf{类型} & \textbf{API} & \textbf{适用场景} \\
\midrule
Event & \texttt{ReadEvent/WriteEvent} & CAN 消息、Modbus 帧 \\
Stream & \texttt{ReadStream/WriteStream} & 串口 UART 数据流 \\
Snapshot & \texttt{ReadSnapshot/WriteSnapshot} & GPIO 状态、模拟量 \\
\bottomrule
\end{tabularx}
\caption{硬件 IO 抽象层}
\end{table}

\subsection{PC 端模拟调试}

无需实际硬件即可进行开发测试,大幅提升调试效率:

\begin{lstlisting}[language={[Sharp]C}, caption={PC 端模拟调试}]
// 继承 LocalDebugDIVERVehicle 即可在 PC 上运行
public class MyVehicle : LocalDebugDIVERVehicle
{
    [AsLowerIO] public int sensorValue;
    [AsUpperIO] public int motorSpeed;
}

// 运行测试
var vehicle = new MyVehicle();
vehicle.RunDIVER();  // 在 PC 上模拟 MCU 执行
\end{lstlisting}

\subsection{源码级错误定位}

当 MCU 发生崩溃时,DIVER 能够精确定位到 C\# 源代码位置:

\begin{lstlisting}[language=json, caption={源码映射示例}]
{
  "ilOffset": 239,
  "methodName": "MotorControl.Operation(Int32)",
  "sourceFile": "MotorControl.cs",
  "sourceLine": 15
}
\end{lstlisting}

\noindent 错误处理流程:
\begin{enumerate}[noitemsep]
    \item 捕获错误位置(IL 偏移)
    \item 查询源码映射文件
    \item 输出 C\# 文件名和行号
\end{enumerate}

\subsection{自定义原生扩展}

当需要高性能计算或特殊硬件访问时,可以用 C 语言编写原生函数,在 C\# 中直接调用。编译器会自动桥接两种语言:

\begin{lstlisting}[language=C, caption={C 端实现高性能函数}]
// 在 MCU 端用 C 实现(additional_builtins.h)
void builtin_FastSqrt(uchar** reptr) {
    int value = pop_int(reptr);      // 从栈获取参数
    int result = fast_isqrt(value);  // 高效整数开方
    push_int(reptr, result);         // 返回结果
}
\end{lstlisting}

\begin{lstlisting}[language={[Sharp]C}, caption={C\# 中直接调用}]
// C# 逻辑代码中直接调用,编译器自动桥接
public override void Operation(int iteration)
{
    int distance = cart.sensor_distance;
    int sqrt_dist = FastMath.FastSqrt(distance);  // 调用 C 原生函数
    cart.output = sqrt_dist * 10;
}
\end{lstlisting}

\noindent 这种机制让开发者可以:
\begin{itemize}[noitemsep]
    \item 用 C 实现性能关键的算法(如 PID、滤波、FFT)
    \item 访问特殊硬件寄存器
    \item 复用现有的 C 库代码
\end{itemize}

\subsection{Web 控制面板}

CoralinkerHost 提供基于 ASP.NET + Vue 3 的可视化管理界面:

\begin{itemize}[noitemsep]
    \item \textbf{节点管理}:添加/删除/配置 MCU 节点
    \item \textbf{变量监视}:实时 UpperIO/LowerIO 数值显示
    \item \textbf{在线编译}:Web IDE 编辑代码并一键部署
    \item \textbf{错误诊断}:致命错误弹窗,支持跳转源码
\end{itemize}

%========================================
\section{适用场景}
%========================================

\subsection{机器人控制}

DIVER 特别适合机器人底层控制开发:
\begin{itemize}[noitemsep]
    \item 多轴电机协调控制(CANOpen、Modbus)
    \item 传感器数据采集与融合
    \item 安全保护逻辑(急停、触边检测)
\end{itemize}

\begin{lstlisting}[language={[Sharp]C}, caption={差速底盘控制示例}]
public override void Operation(int iteration)
{
    // 安全检查
    if (cart.emergency_stop == 1 || cart.edge_touch == 1)
    {
        EmergencyStopAllMotors();
        return;
    }
    
    // 差速计算
    int leftSpeed = cart.linear_vel - cart.angular_vel;
    int rightSpeed = cart.linear_vel + cart.angular_vel;
    
    // 发送电机命令
    motors[LEFT].TargetSpeed = leftSpeed;
    motors[RIGHT].TargetSpeed = rightSpeed;
}
\end{lstlisting}

\subsection{工业设备控制}

\begin{itemize}[noitemsep]
    \item PLC 类梯形图逻辑实现
    \item Modbus RTU/TCP 设备通信
    \item 多节点分布式控制系统
\end{itemize}

\subsection{智能硬件原型}

\begin{itemize}[noitemsep]
    \item 快速验证控制算法
    \item 传感器数据采集与处理
    \item 执行器驱动开发
\end{itemize}

\subsection{教育与培训}

\begin{itemize}[noitemsep]
    \item 嵌入式系统教学实验
    \item 控制理论实践平台
    \item 机器人竞赛开发环境
\end{itemize}

%========================================
\section{技术规格}
%========================================

\begin{table}[htbp]
\centering
\renewcommand{\arraystretch}{1.3}
\begin{tabularx}{\textwidth}{>{\bfseries}l X}
\toprule
项目 & 规格 \\
\midrule
主机端要求 & .NET 8.0+,Windows / Linux \\
MCU 端要求 & ARM Cortex-M 系列,$\geq$512KB Flash,$\geq$128KB RAM \\
通信接口 & 串口(1M--2Mbps)、以太网 \\
扫描周期 & 可配置,典型值 10--50ms \\
IO 支持 & CAN、RS485/Modbus、GPIO、ADC、UART \\
编译输出 & 字节码(.bin)+ 源码映射 + 变量描述符 \\
\bottomrule
\end{tabularx}
\caption{DIVER 技术规格}
\end{table}

%========================================
\section{项目结构}
%========================================

\begin{table}[htbp]
\centering
\renewcommand{\arraystretch}{1.2}
\begin{tabularx}{\textwidth}{l X}
\toprule
\textbf{目录} & \textbf{说明} \\
\midrule
\texttt{DiverCompiler/} & C\# → MCU 字节码编译器 \\
\texttt{MCURuntime/} & MCU 端 C 运行时 \\
\texttt{DiverTest/} & 测试框架与示例代码 \\
\texttt{MCUSerialBridge/} & 串口通信协议栈 \\
\texttt{3rd/CoralinkerHost/} & Web 控制面板(ASP.NET + Vue 3) \\
\texttt{3rd/CoralinkerSDK/} & 多节点管理 SDK \\
\bottomrule
\end{tabularx}
\caption{项目目录结构}
\end{table}

%========================================
\vfill
\begin{center}
\begin{tikzpicture}
    \node[draw=diverblue, line width=2pt, rounded corners=8pt, 
          minimum width=14cm, minimum height=1.2cm, 
          fill=diverblue!10] {
        \Large\bfseries DIVER —— 让嵌入式开发更简单、更高效、更可靠
    };
\end{tikzpicture}
\end{center}

\end{document}
